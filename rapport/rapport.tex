\documentclass[a4paper]{article}

\usepackage{geometry}
\usepackage[francais]{babel}
\usepackage[utf8]{inputenc}
\usepackage{graphicx}
\usepackage[T1]{fontenc}
\usepackage{listings}

\title{Projet de Machine Learning - Spaceship Titanic}
\author{Antoine ROUMILHAC \\ Alban RIO \\ Nabil DJELLOUDI \\ Mohammed Yacine BRAHMIA}
\date{Mars 2024}

\newcommand{\illustration}[3]{
    \begin{figure}[h!]
        \centering
        \includegraphics[width=#3]{#1}
        \caption{#2}
    \end{figure}
}

\begin{document}
    \maketitle

    Nom de l'équipe Kaggle : On coule

    \section{Analyse des données}

    \subsection{Répartition passagers transportés / non transportés}

    Dans les données d'entraînement, 4378 des passagers ont été transportés, contre 4315 qui ne l'ont pas été, donc l'ensemble est équilibré. On n'aura donc pas à ajouter de poids lors de la phase d'entraînement.

    \subsection{Valeurs manquantes}

    
    Sur la Figure 1, on peut observer qu'il y a des données manquantes dans l'ensemble d'entraînement, mais qu'elles sont globalement réparties entre les différents passagers.
    On va donc pouvoir remplir les données manquantes en exploitant les relations entre les données.
    
    \illustration{images/Figure 1.png}{Heatmap des données manquantes sur l'ensemble d'entraînement}{8cm}
    \subsection{Variables intéressantes}

    La variable CryoSleep semble être très intéressante à utiliser, car comme présenté sur la figure 2, les personnes ayant pris le CryoSleep n'ont très majoritairement pas été transportées,
    et inversement les personnes ayant pris le CryoSleep ont très majoritairement été transportées.
    
    \illustration{images/Figure 2.png}{Répartition des personnes selon le cryosleep et leur état de transport}{8cm}
    
    On peut voir sur la figure 3 que les personnes ayant entre 0 et 12 ans ont été transportées en majorité, 
    tandis que celles ayant entre 18 et 25 ans, et entre 30 et 40 ans, ont été peu moins transportées que la moyenne.
    Il serait donc intéressant d'extraire ces tranches d'âge pour les utiliser pour l'entraînement.

    \illustration{images/Figure 3.png}{Répartition des personnes selon leur âge et leur état de transport}{8cm}
    
    \illustration{images/Figure 4.png}{Répartition des personnes selon leur planète et leur état de transport}{8cm}
    
    \illustration{images/Figure 5.png}{Répartition des personnes selon leur destination et leur état de transport}{8cm}
    
    \illustration{images/Figure 6.png}{Répartition des personnes selon leurs dépenses dans le room service et leur état de transport}{8cm}

\end{document}